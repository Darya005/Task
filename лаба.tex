\documentclass{article}
\usepackage[T2A]{fontenc}
\usepackage{amsthm}
\usepackage{amsmath}
\usepackage{amssymb}
\usepackage{amsfonts}
\usepackage{mathrsfs}
\usepackage[12pt]{extsizes}
\usepackage{fancyvrb}
\usepackage{indentfirst}
\usepackage{graphicx}
\graphicspath{{pictures/}}
\DeclareGraphicsExtensions{.pdf,.png,.jpg}
\usepackage[
  left=2cm, right=2cm, top=2cm, bottom=2cm, headsep=0.2cm, footskip=0.6cm, bindingoffset=0cm
]{geometry}
\usepackage[russian]{babel}

\begin{document}

\author{Пичугина Дарья, Михайлин Дмитрий 151 группа}
\title{Лабораторная работа №2}
\maketitle

\begin{flushright}
    Проверил доцент, к.,ф."=м.,н. \qquad\qquad  \qquad\qquad \qquad\qquad В.,В.,Машников
\end{flushright}

\textbf{Наименование работы:} Проверка основного уравнения 
вращательного движения с помощью маятника Обербека.

\textbf{Цель работы:} опытная проверка основного уравнения 
вращательного движения, оценка точности метода измерения.

\textbf{Принадлежности:} маятник Обербека, секундомер,
штангенциркуль, линейка, набор гирь и разновесок.

\section{Рабочая фомула}

\begin{equation}
    \frac{md}{2}\Big(g-\frac{2h}{t^2}\Big) = \frac{4h}{dt^2}I
\end{equation}
\vspace{1cm}

Основное уравнение вращательного движения: 
\begin{equation}
    \bar{M}_z=I_z\frac{d\omega}{dt} 
\end{equation}

На оси прибора (рис. 2) укреплен вал радиуса $r$ и крестовина
$K$. Вдоль стержней крестовины могут перемещаться грузики,
которые можно закреплять в нужных положениях. На вал намотана
нить, к которой прикреплено тело массой $m$, приводящее при
своем падении вал во вращение.

\includegraphics[scale=0.5]{1}

Уравнение движения тела запишется в виде:
\begin{equation}
    mg - F_H = ma
\end{equation}

Сила, по величине равная $F_H$, но направленная противоположно,
также действует на вал, создавая вращательный момент
\begin{equation}
    M = rF_H = rm(g-a)
\end{equation}

заставляющий вал вращаться с угловым ускорением
$\frac{d\omega}{dt}$.

Согласно основному уравнению вращательного движения следует
записать:
\begin{equation}
    rm(g-a) = I\frac{d\omega}{dt} 
\end{equation}
где $I$ "--- момент инерции вращающейся системы.

Расстояние, пройденное телом во время падения и
ускорение тела:
\begin{equation}
    h = \frac{at^2}{2}~\text{и}~a = \frac{2h}{t^2}
\end{equation}

Угловое ускорение связано с ускорением соотношением: 
\begin{equation}
    \frac{d\omega}{dt} = \frac{a}{r} = \frac{2h}{rt^2}
\end{equation}

Подставим значения $a$ и $\frac{d\omega}{dt}$ в (4) и получим:
\begin{equation}
    rm\Big(g-\frac{2h}{t^2}\Big) = I\frac{2h}{t^2}
\end{equation}

Учитывая, что $r=\frac{d}{2}$, где $d$ "--- диаметр вала,
окончательно имеем:
\begin{equation}
    \frac{md}{2}\Big(g-\frac{2h}{t^2}\Big) = \frac{4h}{dt^2}I
\end{equation}

Это уравнение и подлежит экспериментальной проверке. Если
закон справедлив, то значения моментов  в левой и правой
частях уравнения должны совпадать в пределах погрешности.

\section{Ход работы}
\textbf{1.} Установим грузики у самых концов стержней маятника таким
образом, чтобы маятник находился в безразлином состоянии.

\textbf{2.} Намотаем нить на вал и, отпустив груз $m$, определим по
секундомеру время $t$ его падения на всю длину нити.

\textbf{3.} Проделаем опыт несколько раз и определим средее время
падения $t$.

\textbf{4.} Измерим линейкой высоту падения $h$.

\textbf{5.} Определим массу груза $m$.

\textbf{6.} Измерим штангенциркулем диаметр вала $d$.

\textbf{7.} Вычислим значение момента силы $M_1$, определяемого левой
частью уравнения (1).

\textbf{8.} Рассчитаем значение момента силы $M_2$, определяемого
правой частью уравнения (1). Значение момента инерции $I$
маятника указано на установке.

\textbf{9.} Полученные значения величин занесем в таблицу 1:

\includegraphics[scale=0.5]{2}

\textbf{10.} Проведем расчет погрешности измерений. Пусть абсолютная
погрешность в вычислении момента силы $M$ первым способом
$\Delta M_1$, а вторым "--- $\Delta M_2$, то есть $M=M_1+
\Delta M_1$ и $M=M_2\pm\Delta M_2$. Значение момента силы
можно считать совпадающим в данном эксперименте, если
выполняется условие 
\begin{equation}
    \left\lvert M_1 - M_2 \right\rvert \leq
    \left\lvert \Delta M_1 \right\rvert + 
    \left\lvert \Delta M_2 \right\rvert
\end{equation}

Значение погрешностей $\Delta M_1$ и $\Delta M_2$ можно
определить следующим путем. Прологарифмировав и
продифференцировав левую часть уравнения (1), получим значение
относительной погрешности:
\begin{equation}
    \delta_1 = \frac{\Delta M_1}{M_1} = \frac{\Delta m}{m}
    + \frac{\Delta d}{d} + 2~\frac{t\Delta h +
    h\Delta t}{t(gt^2-2h)}
\end{equation}

Значение абсолютной погрешности будет иметь вид $\Delta M_1$
= $\delta M_1$. Аналогичным образом имеем
\begin{equation}
    \delta_2 = \frac{\Delta M_2}{M_2} = \frac{\Delta I}{I}
    + \frac{\Delta h}{h} + \frac{\Delta d}{d} +
    2~\frac{\Delta t}{t} \text{;}~ \Delta M_2 = \delta_2 M_2
\end{equation}

За погрешности $\Delta d$, $\Delta m$ и $\Delta h$ принимаем
погрешности отсчитывания соответствующих средств измерений.
В качестве погрешности $\Delta t$ рассматриваем среднюю 
абсолютную погрешность результата измерения, если ее 
значение превышает погрешность отсчитывания секундомера.

\textbf{11.} Значения моментов $M_1$ и $M_2$ приблизительно
совпадают, значит закон справедлив.

Зависимость $\frac{d\omega}{dt}=\frac{4h}{dt^2}=\beta$ от 
величины $M_1$ отобразим на графике.

\vspace{0.6cm}

\includegraphics[scale=0.28]{3}

\vspace{0.6cm}

Тогда момент инерции будет равен: $I=\tan{\alpha}$;
$~ I\approx 0,04~\text{кг}\cdot \text{м}^2$.

Полученное значение момента инерции примерно равно значению,
приведенному на установке.

\textbf{12.} 
в ходе проделанной работы мы изучили 

\end{document}