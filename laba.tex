\documentclass{article}
\usepackage[T2A]{fontenc}
\usepackage{amsthm}
\usepackage{amsmath}
\usepackage{amssymb}
\usepackage{amsfonts}
\usepackage{mathrsfs}
\usepackage[12pt]{extsizes}
\usepackage{fancyvrb}
\usepackage{indentfirst}
\usepackage{graphicx}
\graphicspath{{pictures/}}
\DeclareGraphicsExtensions{.pdf,.png,.jpg}
\usepackage[
  left=2cm, right=2cm, top=2cm, bottom=2cm, headsep=0.2cm, footskip=0.6cm, bindingoffset=0cm
]{geometry}
\usepackage[russian]{babel}

\begin{document}

\author{Пичугина Дарья, Михайлин Дмитрий 151 группа}
\title{Лабораторная работа №2}
\maketitle

\begin{flushright}
    Проверил доцент, к.,ф."=м.,н. \qquad\qquad  \qquad\qquad \qquad\qquad В.В.Машников
\end{flushright}

\textbf{Наименование работы:} Проверка основного уравнения 
вращательного движения с помощью маятника Обербека.

\textbf{Цель работы:} опытная проверка основного уравнения 
вращательного движения, оценка точности метода измерения.

\textbf{Принадлежности:} маятник Обербека, секундомер,
штангенциркуль, линейка, набор гирь и разновесок.

\section{Рабочая фомула}

\begin{equation}
    \frac{md}{2}\Big(g-\frac{2h}{t^2}\Big) = \frac{4h}{dt^2}I
\end{equation}
\vspace{1cm}

Основное уравнение вращательного движения: 
\begin{equation}
    \bar{M}_z=I_z\frac{d\omega}{dt} \nonumber
\end{equation}

На оси прибора (рис. 2) укреплен вал радиуса $r$ и крестовина
$K$. Вдоль стержней крестовины могут перемещаться грузики,
которые можно закреплять в нужных положениях. На вал намотана
нить, к которой прикреплено тело массой $m$, приводящее при
своем падении вал во вращение.

\includegraphics[scale=0.5]{1}

Уравнение движения тела запишется в виде:
\begin{equation}
    mg - F_H = ma \nonumber
\end{equation}

Сила, по величине равная $F_H$, но направленная противоположно,
также действует на вал, создавая вращательный момент
\begin{equation}
    M = rF_H = rm(g-a) \nonumber
\end{equation}

заставляющий вал вращаться с угловым ускорением
$\frac{d\omega}{dt}$.

Согласно основному уравнению вращательного движения следует
записать:
\begin{equation}
    rm(g-a) = I\frac{d\omega}{dt}  \nonumber
\end{equation}
где $I$ "--- момент инерции вращающейся системы.

Расстояние, пройденное телом во время падения и
ускорение тела:
\begin{equation}
    h = \frac{at^2}{2}~\text{и}~a = \frac{2h}{t^2} \nonumber
\end{equation}

Угловое ускорение связано с ускорением соотношением: 
\begin{equation}
    \frac{d\omega}{dt} = \frac{a}{r} = \frac{2h}{rt^2} \nonumber
\end{equation}

Подставим значения $a$ и $\frac{d\omega}{dt}$ в (4) и получим:
\begin{equation}
    rm\Big(g-\frac{2h}{t^2}\Big) = I\frac{2h}{t^2} \nonumber
\end{equation}

Учитывая, что $r=\frac{d}{2}$, где $d$ "--- диаметр вала,
окончательно имеем:
\begin{equation}
    \frac{md}{2}\Big(g-\frac{2h}{t^2}\Big) = \frac{4h}{dt^2}I \nonumber
\end{equation}

Это уравнение и подлежит экспериментальной проверке. Если
закон справедлив, то значения моментов  в левой и правой
частях уравнения должны совпадать в пределах погрешности.

\section{Ход работы}
\textbf{1.} Установим грузики у самых концов стержней маятника таким
образом, чтобы маятник находился в безразлином состоянии.

\vspace{0.5cm}

\textbf{2.} Намотаем нить на вал и, отпустив груз $m$, определим по
секундомеру время $t$ его падения на всю длину нити.

\vspace{0.5cm}

\textbf{3.} Проделаем опыт несколько раз и определим средее время
падения $t$.

\vspace{0.5cm}

\textbf{4.} Измерим линейкой высоту падения $h$.

\vspace{0.5cm}

\textbf{5.} Определим массу груза $m$.

\vspace{0.5cm}

\textbf{6.} Измерим штангенциркулем диаметр вала $d$.

\vspace{0.5cm}

\textbf{7.} Вычислим значение момента силы $M_1$, определяемого левой
частью уравнения (1).

\vspace{0.5cm}

\textbf{8.} Рассчитаем значение момента силы $M_2$, определяемого
правой частью уравнения (1). Значение момента инерции $I$
маятника указано на установке.

\vspace{0.5cm}

\textbf{9.} Полученные значения величин занесем в таблицу 1:

\includegraphics[scale=0.5]{2}

Значения моментов $M_1$ и $M_2$ приблизительно
совпадают, значит закон справедлив.

\vspace{0.5cm}

\textbf{10.} Зависимость $\frac{d\omega}{dt}=\frac{4h}{dt^2}=\beta$ от 
величины $M_1$ отобразим на графике.

\vspace{0.6cm}

\includegraphics[scale=0.28]{3}

\vspace{0.6cm}

Тогда момент инерции будет равен: $I=\tan{\alpha}$;
$~ I\approx 0,038~\text{кг}\cdot \text{м}^2$.

Полученное значение момента инерции примерно равно значению,
приведенному на установке.

\textbf{Анализ полученных результатов.} 
В ходе проделанной работы мы выяснили, что значения
$M_1$ и $M_2$ приблизительно совпадают, поэтому можно сделать вывод, 
что закон справедлив.

\section{Дополнительное задание}

1. Время падения системы без  цилиндров: $t=1,082$ с.

2. Диаметр цилиндра: $d=4$ см.

3. Расстоние от центра до оси вращения: $a=25$ см.

4. Масса: $m=155$ г.

5. Длина образующей: $l=2$ см.

Момент инерции системы с цилиндрами:

\begin{equation}
    I_\text{сис} = \frac{mR^2(gt^2-2h)}{2h} ~
    \text{или} ~ I_\text{сис} = I_0+4I_\text{ц}~\text{, где}
    ~I_0=0,002 ~ \text{кг} \cdot \text{м}^2 \nonumber
\end{equation}

Вычислим значение момента инерции системы:

\begin{equation}
    I_\text{сис}= 0,038~\text{кг} \cdot \text{м}^2 \nonumber
\end{equation}

Найдем $I_\text{ц}$:

\begin{equation}
    I_\text{ц} = 0,009~\text{кг} \cdot \text{м}^2 \nonumber
\end{equation}

Сравним с теоретическим значением:

\begin{equation}
    I_\text{ц. теор}=\frac{1}{12}m_0l^2+\frac{1}{4}m_0r^2+m_0a^2
    \Rightarrow  I_\text{ц. теор}= 0.0097~\text{кг} \cdot \text{м}^2 \nonumber
\end{equation}

\begin{equation}
    I_\text{ц} \thickapprox I_\text{ц. теор}. \nonumber
\end{equation}
\end{document}