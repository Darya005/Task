\documentclass{article}
\usepackage[english,russian]{babel}

\begin{document}

\title{Глоссарий}
\date{} 
\author{Студентки 151 группы Пичугиной Дарьи}
\maketitle

\section*{Русский алфавит}
\section*{Б}

\hspace{0.33cm} 1. \textbf{Багфиксинг} "--- процесс устранения неисправностей \textit{(Сорокин
Леонид)}.

2. \textbf{Биткоин} (англ. Bitcoin, bit "--- <<бит>> и coin "--- <<монета>>)
"--- пиринговая платёжная система, использующая одноимённую единицу для учёта
операций. Для обеспечения функционирования и защиты системы используются
криптографические методы, но при этом вся информация о транзакциях между
адресами системы доступна в открытом виде \textit{(Юрин Игорь)}.

\section*{В}

\hspace{0.33cm} 3. \textbf{Воркшопы} "--- обучение работе, своего рода курсы \textit{(Акутин
Артём)}.

\section*{Г}

\hspace{0.33cm} 4. \textbf{Глубокое обучение} (англ. deep learning) "--- совокупность методов
машинного обучения, основанных на обучении представлениям, а не
специализированных алгоритмах под конкретные задачи \textit{(Александр
Кузьмин)}.

\section*{Д}

\hspace{0.33cm} 5. \textbf{Дата-майнинг} "--- интеллектуальный анализ данных с целью выявления
закономерностей, технологии обнаружения в сырых данных ранее неизвестных и
практически полезных знаний \textit{(Кузьмин Алексей)}.

6. \textbf{Джун} "--- начинающий программист \textit{(Гаязова Александра)}.

\section*{И}

\hspace{0.33cm} 7. \textbf{Интернет вещей} (англ. internet of things) "---
концепция сети передачи данных между физическими объектами (<<вещами>>),
оснащёнными встроенными средствами и технологиями для взаимодействия друг с
другом или с внешней средой \textit{(Юрин Игорь)}.

8. \textbf{Ивент} "--- мероприятие \textit{(Агеева Ксения)}.

9. \textbf{Интернатура} "--- стажировка джунов \textit{(Агеева Ксения)}.

\section*{К}

\hspace{0.37cm} 10. \textbf{Капсула} "--- объект, использующийся для создания коллизии
\textit{(Юрин Игорь)}.

11. \textbf{Компьютерная безопасность} "--- меры безопасности, применяемые для
защиты вычислительных устройств (компьютеры, смартфоны и другие), а также
компьютерных сетей (частных и публичных сетей, включая Интернет) \textit{(Юрин
Игорь)}.
 

12. \textbf{Кровавый энтерпрайз} "--- среда, в которой дискомфортно работать
\textit{(Соломина Юлия)}.

\section*{М}

\hspace{0.42cm} 13. \textbf{Ментр} "--- наставик у программистов-стажёров \textit{(Гаязова
Александра)}.

14. \textbf{Методология} "--- то, через какие этапы нужно пройти, чтобы наша работа
соответствовала нашим критериям.

15. \textbf{Мидл} "--- программист с опытом \textit{(Гаязова Александра)}.

16. \textbf{Митинг} "--- с англ. <<встреча>> \textit{(Гаязова Александра)}.

17. \textbf{Модульное тестирование} (англ. unit testing) "--- процесс в
программировании, позволяющий проверить на корректность отдельные модули
исходного кода программы, наборы из одного или более программных модулей вместе
с соответствующими управляющими данными, процедурами использования и обработки
\textit{(Елена Ивахнова)}.

\section*{О}

\hspace{0.42cm} 18. \textbf{Облачные вычисления} (англ. cloud computing, акр. от
<<False>>) "--- модель обеспечения удобного сетевого доступа по требованию к
некоторому общему фонду конфигурируемых вычислительных ресурсов (например, сетям
передачи данных, серверам, устройствам хранения данных, приложениям и сервисам —
как вместе, так и по отдельности), которые могут быть оперативно предоставлены и
освобождены с минимальными эксплуатационными затратами или обращениями к
провайдеру \textit{(Александр Кузьмин)}.

\section*{П}

\hspace{0.37cm} 19. \textbf{Прогнозирование} (ML) "--- задача, использующая предыдущие данные
временных рядов, чтобы делать прогнозы о будущем поведении. Сценарии, применимые
к прогнозированию, включают прогнозирование погоды, сезонные прогнозы продаж и
диагностическое обслуживание \textit{(Александр Кузьмин)}.

\section*{Р}

\hspace{0.37cm} 20. \textbf{Ранжирование} (ML) "--- задача, создающая средство ранжирования на
основе набора примеров с метками. Этот набор примеров состоит из групп
экземпляров, которые могут быть оценены с заданными критериями
\textit{(Александр Кузьмин)}.

\section*{С}

\hspace{0.42cm} 21. \textbf{Сеньор} "--- программист"=эксперт \textit{(Гаязова
Александра)}.

22. \textbf{Скилы} "--- навыки \textit{(Высоцкий Александр)}.

23. \textbf{Смерджить} "--- соединить несколько веток \textit{(Рыданов Никита)}.

24. \textbf{Софт скилл} "--- общие знания ЯП \textit{(Акутин Артём)}.

25. \textbf{Стартап} (англ. startup company, startup "--- <<стартующий>>) "---
коммерческий проект, основанный на какой-либо идее и требующий финансирования
для развития \textit{(Ксения Агеева)}.

26. \textbf{Стелс} "--- режим <<скрытность>>; жанр игр, в котором этот режим
присутствует в большом количестве \textit{(Сорокин Леонид)}.

\section*{Т}

\hspace{0.37cm} 27. \textbf{Троян} "--- троянский вирус \textit{(Юрин Игорь)}.

\section*{У}

\hspace{0.37cm} 28. \textbf{Утилита} "--- вспомогательная компьютерная программа \textit{(Юрин
Игорь)}.

\section*{Ф}

\hspace{0.4cm} 29. \textbf{Фидбек} "--- обратная связь.

30. \textbf{Форматирование диска} "--- программный процесс разметки области
хранения данных электронных носителей информации, расположенной на магнитной
поверхности (жёсткие диски, дискеты), оптических носителях
(CD/DVD/Blu"=ray"=диски), твердотельных накопителях (флэш"=память "+-- flash
module, SSD) и др \textit{(Юрин Игорь)}.

\section*{Х}

\hspace{0.37cm} 31. \textbf{Хард скилл} "--- углублённый навык знания ЯП \textit{(Акутин
Артём)}.

\section*{Ж}

\hspace{0.37cm} 32. \textbf{Жучки} "--- устройства слежения \textit{(Юрин Игорь)}.





\section*{Английский алфавит}
\section*{A}

\hspace{0.42cm} 33. \textbf{Amber} "--- это скриптовый язык с открытым исходным кодом, который
разрабатывается для легкой реализации высокоуровневого программирования с
саморасширением \textit{(Ксения Агеева)}.

34. \textbf{AnimNotify} "--- прерывание анимации \textit{(Сорокин Леонид)}.

35. \textbf{API} (Application Programming Interface или интерфейс
программирования приложений) "--- это совокупность инструментов и функций в виде
интерфейса для создания новых приложений, благодаря которому одна программа
будет взаимодействовать с другой. Это позволяет разработчикам расширять
функциональность своего продукта и связывать его с другими.

\section*{B}

\hspace{0.42cm} 36. \textbf{Backend} "--- внутренняя и вычислительная логика веб"=сайта или
веб"=приложения \textit{(Агеева Ксения)}.

37. \textbf{Benefit} (с англ. <<льгота>>, <<выгода>>) "--- дополнительные бонусы
от компании сотрудникам \textit{(Ксения Агеева)}.

38. \textbf{Business intelligence} "--- обозначение компьютерных методов и
инструментов для организаций, обеспечивающих перевод транзакционной деловой
информации в человекочитаемую форму, а также средства для массовой работы с
такой обработанной информацией \textit{(Александр Кузьмин)}.

\section*{C}

\hspace{0.33cm} 39. \textbf{Capacity} (с англ. <<вместимость>>) "--- размер
команды \textit{(Целищев Александр)}.

\section*{D}

\hspace{0.42cm} 40. \textbf{Data mining} "--- собирательное название, используемое для
обозначения совокупности методов обнаружения в данных ранее неизвестных,
нетривиальных, практически полезных и доступных интерпретации знаний,
необходимых для принятия решений в различных сферах человеческой деятельности
\textit{(Александр Кузьмин)}.

41. \textbf{Data science} "--- наука о данных, объединяющая разные области
знаний: информатику, математику и системный анализ. Сюда входят методы обработки
больших данных (Big Data), интеллектуального анализа данных (Data Mining),
статистические методы, методы искусственного интеллекта, в т. ч. машинное
обучение (Machine Learning) \textit{(Ксения Агеева)}.

42. \textbf{DevOps} (акр. от «development) "--- методология автоматизации
технологических процессов сборки, настройки и развёртывания программного
обеспечения. Методология предполагает активное взаимодействие специалистов по
разработке со специалистами по информационно-технологическому обслуживанию и
взаимную интеграцию их технологических процессов друг в друга для обеспечения
высокого качества программного продукта \textit{(Ксения Агеева)}.

\section*{F}

\hspace{0.33cm} 43. \textbf{Framework} "--- программная платформа, определяющая структуру
программной системы; программное обеспечение, облегчающее разработку и
объединение разных компонентов большого программного проекта \textit{(Ксения
Агеева)}.

44. \textbf{Framework} "--- платформа, определяющая структуру программной
системы \textit{(Акутин Артём)}.

45. \textbf{Frontend} "--- клиентская сторона пользовательского интерфейса к
программно"=аппаратной части сервиса \textit{(Агеева Ксения)}.

\section*{G}

\hspace{0.33cm} 46. \textbf{Groovy} "--- объектно-ориентированный язык программирования,
разработанный для платформы Java как дополнение к языку Java с возможностями
Python, Ruby и Smalltalk \textit{(Юлия Соломина)}.

\section*{I}

\hspace{0.33cm} 47. \textbf{IT-community} "--- сообщество работников IT"=сферы \textit{(Гаязова
Александра)}.

\section*{M}

\hspace{0.33cm} 48. \textbf{Machine Learning} "--- машинное обучение \textit{(Кузьмин Алексей)}.

\section*{P}

\hspace{0.33cm} 49. \textbf{Performance review} "--- процесс подведения итогов работы за полгода
\textit{(Агеева Ксения)}.

\section*{Q}

\hspace{0.33cm} 50. \textbf{QA-инженер} (акр. от <<Quality Assurance Engineer>>)
"--- специалист, задача которого заключается в контроле за правильностью
выполнения всех этапов разработки и правильностью работы итогового продукта
\textit{(Ксения Агеева)}.

\section*{S}

\hspace{0.33cm} 51. \textbf{Sprint} "--- способ последовательного выполнения задач
\textit{(Целищев Александр)}.



\end{document}