\documentclass[bachelor, och, referat]{SCWorks}

\usepackage{graphicx}

\usepackage[sort,compress]{cite}
\usepackage{amsmath}
\usepackage{amssymb}
\usepackage{amsthm}
\usepackage{fancyvrb}
\usepackage{longtable}
\usepackage{array}
\usepackage[english,russian]{babel}
\usepackage{tempora}
\usepackage{minted}
\usepackage[colorlinks=true, citecolor=black, filecolor=black, linkcolor=black,
urlcolor=black]{hyperref}
%\usepackage{times}



\begin{document}

% Кафедра (в родительном падеже)
\chair{информатики и программирования}

% Тема работы
\title{РОБОТОТЕХНИКА}

% Курс
\course{1}

% Группа
\group{151}

% Факультет (в родительном падеже) (по умолчанию "факультета КНиИТ")
%\department{факультета КНиИТ}

% Специальность/направление код - наименование
%\napravlenie{02.03.02 "--- Фундаментальная информатика и информационные технологии}
%\napravlenie{02.03.01 "--- Математическое обеспечение и администрирование информационных систем}
%\napravlenie{09.03.01 "--- Информатика и вычислительная техника}
\napravlenie{09.03.04 "--- Программная инженерия}
%\napravlenie{10.05.01 "--- Компьютерная безопасность}

% Для студентки. Для работы студента следующая команда не нужна.
\studenttitle{студентов}

% Фамилия, имя, отчество в родительном падеже
\author{Пичугиной Дарьи Николаевны и Михайлина Дмитрия Константиновича}

% Заведующий кафедрой
\chtitle{к.\,ф.-м.\,н.} % степень, звание
\chname{С.\,В.\,Миронов}

%Научный руководитель (для реферата преподаватель проверяющий работу)
\satitle{доцент, к.\,ф.-м.\,н.} %должность, степень, звание
\saname{А.\,П.\,Грецова}

% Руководитель практики от организации (только для практики,
% для остальных типов работ не используется)
%\patitle{к.\,ф.-м.\,н., доцент}
%\paname{Д.\,Ю.\,Петров}

% Семестр (только для практики, для остальных
% типов работ не используется)
%\term{2}

% Наименование практики (только для практики, для остальных
% типов работ не используется)
%\practtype{учебная}

% Продолжительность практики (количество недель) (только для практики,
% для остальных типов работ не используется)
%\duration{2}

% Даты начала и окончания практики (только для практики, для остальных
% типов работ не используется)
%\practStart{01.07.2016}
%\practFinish{14.07.2016}

% Год выполнения отчета
\date{2022}

\maketitle

% Включение нумерации рисунков, формул и таблиц по разделам
% (по умолчанию - нумерация сквозная)
% (допускается оба вида нумерации)
%\secNumbering


\tableofcontents



% Раздел "Определения". Может отсутствовать в работе
%\definitions

% Раздел "Определения, обозначения и сокращения". Может отсутствовать в работе.
% Если присутствует, то заменяет собой разделы "Обозначения и сокращения" и "Определения"
%\defabbr

\intro

%В наши  дни  современный человеку трудно представить свою обыденную жизнь без привычных удобств "--– результатов многочисленных достижений науки и техники. 21 век "--- эпоха бесчисленных возможностей, коммуникаций и новых технологий; это такой период человеческой истории, когда с каждым годом жизнь людей значительно облегчается, все процессы механизируются, прилавки супермаркетов заполняются экзотической пищей, в торговых комплексах появляются одежды из новейших материалов, а в гипермаркетах электроники и того дальше, невозможно угнаться за развитием новых изобретений. Все привычное старое стремительно сменяется на необыкновенное, новое, к которому так не просто привыкнуть.

Робототехника "--- отдел прикладной науки, который занимается проектированием,
производством и применением автоматизированных технических систем "--- роботов.
В свою очередь робот "--- это автоматическое устройство, которое осуществляет
производственные, бытовые, военные, медицинские и прочие операции по
определенной программе или под управлением человека.

В 1920 году писатель Карл Чапек использовал термин <<робот>> в своей пьесе
<<Россумские универсальные роботы>>. Само слово происходит означает <<тяжелая
монотонная работа>> или <<каторга>>.

Роботы успешно справляются рутинными заданиями, их чаще всего используют для
выполнения многократно повторяющихся работ, которые могут быть сложными или
опасными для здоровья человека и его жизни. В наши дни миллионы роботов
применяются всех сферах человеческой деятельности. Они используются при
управлении различными видами транспорта, спускаются на дно океана, работают в
космосе, собирают бытовые технические устройства, производят микрочипы,
используются военными, помогают спасателям. Во всех областях люди стараются
создать себе автоматических помощников. 

\textbf{Цель реферата.} Изучить историю развития робототехники, а также рассмотреть виды и типы роботов, сферы их использования в современном мире и узнать перспективы развития.

\section{История развития робототехники}

\subsection{Первые прообразы роботов}

Cо времен античности люди стали задумываться о создании механических людей,
способных выполнять тяжелую или монотонную работу.  В качестве примера можно
привести описанную Филоном Византийским механическую женщину"=слугу, которая
наливала из кувшина вино во вставленный в ее руку стакан. Также в
древнегреческих мифах про Гефеста есть упоминания о механических рабах, которые
выполняли работу за человека. Стоит упомянуть греческого математика и
изобретателя Архита из Тарентума, который еще в 5 веке до н. э. изобрел
деревянного голубя (рис. ~\ref{fig:1}), который запускался в небо с помощью
паровой катапульты. Многие историки технологий считают, что это первый прототип
современных крылатых ракет \cite{1}. 

\begin{figure}[h!]
    \centering
    \includegraphics[width=10cm]{1.png}
    \caption{\label{fig:1}%
    Деревянный голубь Архита Тарентского}
\end{figure}

В древнегреческом городе Сиракузы на острове Сицилия жил великий греческий
 изобретатель и ученый Архимед, также прославившийся созданием автоматических
 механизмов. В частности, ему приписывается создание первого прообраза
 настоящего боевого робота (рис. ~\ref{fig:2}). Устройство под названием
 <<коготь>>, устанавливаемое на крепостной стене, захватывало длинным крюком
 осаждавшие город римские корабли, поднимало их в воздух и переворачивало,
 стряхивая экипаж за борт.
\begin{figure}[h!]
    \centering
    \includegraphics[width=10cm]{2.png}
    \caption{\label{fig:2}%
    Коготь Архимеда}
\end{figure}

\vspace{5cm}

 История робототехники была бы неполной без достижений других государств того
 времени. Так, еще в конце 2 тысячелетия до н. э., задолго до древнегреческих
 механизмов, в Древнем Египте жрецы изготовили статую, которая поднятием руки
 указывала на наследника фараона во время религиозных церемоний. А в Китае
 примерно в это же время местные мастера создавали первые прототипы роботов,
 приводимые в действие силой пороховых взрывов. Великий мудрец Лао"=Цзы упоминал
 о механическом человеке, разработанном для императора на рубеже 1 и 2
 тысячелетия до н. э.

И все же именно Древнюю Грецию можно считать родиной робототехники, потому как
здесь были не просто построены многие автоматические устройства, но
теоретизированы принципы их создания и функционирования.

\subsection{«Роботы» Средневековья}

Несмотря на падение античных империй, многие науки продолжали своё развитие,
хотя и с некоторой задёржкой, в числе этих наук была и механика. Одной из сил,
благодаря которой в средние века появились на свет сложные технологические
устройства, была Церковь. В те времена католические монастыри были одним из
центров научной и инженерной мысли. В частности, легенды приписывают виднейшему
ученому и теологу Альберту Великому создание <<механической служанки>> (рис.
~\ref{fig:3}), которая умела самостоятельно передвигаться и даже воспроизводить
речь. Задокументированным, и, следовательно, более достоверным, выглядит
свидетельство средневекового архитектора Виллара де Онекура (13 век н. э.),
который в своем труде описал зооморфные механизмы, а также фигуру ангела,
поворачивающуюся вслед за движением солнца.
\begin{figure}[h!]
    \centering
    \includegraphics[width=10cm]{3.png}
    \caption{\label{fig:3}%
    "Механическая служанка" Альберта Великого}
\end{figure}


\subsubsection{Робот Леонардо да Винчи}

%Наиболее популярными в средние века были автоматические часовые механизмы и
%человеческие фигуры, которые двигались. Так в 1495 году Леонардо да Винчи
%создает проект механического рыцаря, чтобы показать, что машина может двигаться
%как человек. Но Леонардо так его и не сконструировал. Это изобретение эпохи
%Возрождения считается первым в истории роботом.

Изобретателем одного из первых роботов считается итальянский ученый Леонардо да
Винчи. Судя по документам, обнаруженным в 1950"=е годы, художник разработал
чертеж человекоподобного робота в 1495 году (рис. ~\ref{fig:4}). В схемах был
изображен каркас робота, который был запрограммирован выполнять человеческие
движения. Он обладал анатомически правильной моделью челюсти и умел садиться,
двигать руками и шеей. Записи гласили, что поверх каркаса должна быть надета
рыцарская броня. Скорее всего, идея создать <<искусственного человека>> пришла в
голову художнику в ходе изучения человеческого тела \cite{2}.

\begin{figure}[h!]
    \centering
    \includegraphics[width=8cm]{4.png}
    \caption{\label{fig:4}%
    Робот Леонардо да Винчи}
\end{figure}

К сожалению, ученым не удалось найти подтверждений тому, что робот Леонардо да
Винчи действительно был создан. Скорее всего, идея так и осталась на бумаге и
так и не была воплощена в реальность. Зато робот был воссоздан в современности,
спустя сотни лет после разработки чертежа. Сборкой робота занялся итальянский
профессор Марио Таддей, который считается экспертом по изобретениям Леонардо да
Винчи. При сборке механизма он строго следовал чертежам художника и в конечном
итоге создал то, чего хотел добиться изобретатель. Конечно, широкими
возможностями этот робот не блещет, но зато профессор смог написать книгу
<<Машины Леонардо да Винчи>>, которая была переведена на 20 языков.

\subsection{Автоматоны эпохи Возрождения}

Однако настоящую популярность и бурное развитие автоматические механизмы
получили с началом эпохи Возрождения. Наука, вырвавшись из монополии Церкви,
получила дополнительный импульс к развитию, в том числе за счет переосмысления
достижения античных ученых. И на первую роль в новой волне старинной
робототехники вышли часовщики. Здесь стоит упомянуть о двух важных изобретениях,
которые способствовали развитию технологии автоматонов "--- пружинному и
маятниковому заводным механизмам. До этого подобные устройства приводились в
движение гирями, что позволяло создавать только крупные и относительно несложные
изделия. Новые накопители энергии (пружина и маятник) стали настоящим прорывом в
миниатюризации автоматических механизмов.

Особенно прославился на этом поприще мастер Жак де Вокансон, который жил в 18
веке "--- к слову, в детстве обучавшийся в иезуитской школе. Особенную
популярность получили два его изобретения:

\begin{enumerate}
    \item \textbf{механическая утка}, способная взмахивать крыльями, клевать
    зерно с руки и даже испражняться;
    \item \textbf{автоматический музыкант}, умеющий наигрывать различные мелодии
    на флейте и свирели.
\end{enumerate}

Другим известным мастером был швейцарец Пьер Жаке Дро, живший в том же 18 веке и
основавший знаменитую часовую компанию Jaquet Droz. В то время он прославился не
только своими хронометрами, но и множеством сложнейших устройств, среди которых
особенно известно три его творения (рис. ~\ref{fig:5}):

\begin{enumerate}
    \item \textbf{<<Писарь>>} "--- автоматическая фигура мальчика, содержащая
    около 4 000 деталей, была способна написать любой текст из 40 знаков,
    самостоятельно макая перо в чернильницу;
    \item \textbf{<<Художник>>} "--- похожий автомат, только вместо текста
    наносивший на бумагу различные рисунки, например портреты людей, изображения
    животных и т. д.;
    \item \textbf{<<Девушка"=музыкант>>} "--- автомат в виде органистки, который
    умел наигрывать на небольшом органе 5 различных мелодий, при этом двигая
    головой и телом, а в конце выступления изящно кланяясь.
\end{enumerate}

\begin{figure}[h!]
    \centering
    \includegraphics[width=10cm]{5.png}
    \caption{\label{fig:5}%
    Творения Пьера Жаке Дро}
\end{figure}

Отличительной чертой этих автоматонов была возможность их программировать, для
чего использовались барабаны или диски с насечками, в которых была закодирована
последовательность действий. Поменяв их расположение, мастер мог заставить свои
устройства написать различные тексты, сыграть другую мелодию и т. д. И все же
утверждать, что именно он создал первого робота, нельзя "--- его механизмы еще
слишком мало взаимодействовали с внешней средой, а их функции были сугубо
развлекательными.

Изготовление автоматонов развивалось по пути не только усложнения, но и
миниатюризации устройств. Если первые образцы таких механизмов занимали
достаточно много места, то к 19 веку их часто умещали в карманные часы. В
основном это были сугубо развлекательные устройства, изготавливаемые для
аристократов, передвижных цирков, выставок и т. д. Однако пройдет совсем немного
времени, и автоматы начнут помогать людям.

\subsection{Современный этап развития робототехники}

Механические игрушки"=автоматоны изготавливались часовщиками вплоть до начала 20
столетия. Их главным недостатком был сильно ограниченное время действия и
слабость из"=за особенностей пружинного заводного механизма. Однако развитие
технологии электричества дало человечеству новый источник энергии, которым можно
было питать устройства гораздо более продолжительное время. В то же время
начинаются и первые попытки заставить сложные механизмы работать на человека,
заменяя его труд на производстве. Уже в 1808 году французский ткач Жозеф Мари
Жаккар изобрел ткацкий станок, программируемый с помощью перфокарт (рис.
~\ref{fig:6}). Пока это был еще не робот "--- скорее, аналог современных
автоматизированных линий. Но \textbf{именно в нем впервые в промышленности был
реализован принцип программирования, на котором держится современная
робототехника.}

\begin{figure}[h!]
    \centering
    \includegraphics[width=10cm]{6.png}
    \caption{\label{fig:6}%
    Ткацкий станок Жозефа Мари Жаккара}
\end{figure}

Параллельно совершенствовались и способы управления "--- в частности проводной и радиоволновой. В 1898 году Никола Тесла впервые продемонстрировал самоходную лодку, управляемую дистанционно с помощью радио. Одновременно вместо сложных механических приводов устройства начали обзаводиться более простыми, мощными и миниатюрными электрическими двигателями.

Какое"=то время все созданные роботы предназначались для демонстрации научных
достижений, но не для практической деятельности. Возникновение робототехники в
производстве или сельском хозяйстве произошло позже, потому как такая работа
требовала качественно нового уровня технологий. Полноценное развитие
робототехники в промышленности произошло лишь после окончания Второй мировой
войны.

Патент на изобретение, положившее начало эре промышленной робототехники был
выдан 13 июня 1961 года. Робот, о котором пойдет речь "--- Unimate, выпускаемый
компанией Unimation с 1961 года. Первый экземпляр появился на заводе Inland
Fisher Guide Plant, принадлежавшем General Motors; он стал символом технической
революции своего времени, попал на выставки и даже в телешоу. Детище двух
гениальных инженеров, Джорджа Девола (обладателя патента) и Джозефа
Энгельбергера (его соавтора и бизнес-партнера), во многом опередило свое время \cite{3}.

 Уже к середине 60"=х годов в развитых странах насчитывалось несколько десятков
 компаний, наладивших выпуск подобных машин. Особенно в этом преуспела Япония
 "--- закупив у «Юнимейшн» первые роботы в 1968 году, уже через 10 лет эта
 страна стала мировым лидером по выпуску собственных аналогов и оснащения ими
 производств.

Сегодня роботы проникли практически во все сферы деятельности. Промышленность,
научные исследования, энергетика, медицина, развлечения, военные действия и даже
космос "--- современные автоматические или дистанционно контролируемые механизмы
используются очень широко и даже постепенно вытесняют человеческий труд.
Развитие роботов идет по нескольким направлениям "--- улучшение механизмов и
приводов, совершенствование алгоритмов, внедрение самообучающихся систем
управления (слабого искусственного интеллекта), а также разработка новых
интерфейсов <<человек-компьютер>>. Роботизация тесно переплетается с
биотехнологиями и кибернетикой, результатом чего является создание
кибернетических организмов (киборгов), функциональных бионических протезов,
полностью автономных автомобилей, кораблей, космических и летательных аппаратов
(в том числе военных). 

\section{Поколения роботов}

Роботы различных поколений взаимно дополняют друг друга и находят применение
соответственно своим функциональным возможностям и условиям экономической
целесообразности. К настоящему времени сформировалось три поколения роботов \cite{4}.

\subsection{Роботы первого поколения}

\textbf{Роботы первого поколения} "--- это роботы с программным управлением
(программные роботы), предназначенные для выполнения определенной, жестоко
запрограммированной последовательности операций. Роботы первого поколения
принципиально не могут функционировать автономно в недетерминированной
обстановке (рис ~\ref{fig:7}). К ним относится подавляющее большинство промышленных роботов,
осуществляющих транспортные операции, механическую и термическую обработку,
простейшие сборочные операции, сварку, штамповку и т. д.

\begin{figure}[h!]
    \centering
    \includegraphics[width=10cm]{7.png}
    \caption{\label{fig:7}%
    Робот первого поколения}
\end{figure}

Первые такие роботы были освоены промышленностью в начале и середине 60-х гг. Их
система управления имеет только исполнительный уровень: поскольку окружающие
условия не меняются, сенсорных устройств нет "--- никакой связи с внешним миром
не требуется, и она отсутствует. Конструкционно система управления программного
робота представляет собой, например, штекерную или кнопочную панель (рис.
~\ref{fig:8}), с помощью которой оператор, вставляя в определенные ее гнезда
штекеры (аналоги обычной электрической вилки) или нажимая на кнопки, замыкает
тем самым электрические цепи тех или иных приводов манипулятора и задает ему
нужную последовательность движений.

\begin{figure}[h!]
    \centering
    \includegraphics[width=4.5cm]{8.png}
    \caption{\label{fig:8}%
    Система управления программного робота с кнопочной панелью}
\end{figure}

В современных роботах первого поколения хранителем программ является магнитная
или перфолента.

Очевидно, что даже малейшее изменение окружающей обстановки (другое расположение
или ориентация деталей, отклонение параметров технологического процесса) ведет к
нарушению действия программного робота. Тем не менее создание таких роботов
явилось принципиально важным новшеством в деле автоматизации производства. Ведь
простота изменения программы, возможность <<переобучения>> роботов с целью
выполнения новых операций, позволяет передать автоматам выполнение тяжелых или
утомительных для человека простых действий, а также дает возможность
автоматизировать изготовление несложной мелкой и среднесерийной продукции, т. е.
выпускаемой сравнительно небольшими партиями, когда переходы к выпуску новых
изделий происходят довольно часто.

Широкая эксплуатация роботов первого поколения показала их большую надежность.
Это тоже важное достоинство программных роботов, позволившее им занять в
современном производстве весьма прочные позиции, ибо универсальность и
надежность "--- главные характеристики робототехнических систем, в первую очередь
необходимые для успешного внедрения их на промышленных предприятиях.

К сожалению, функциональные возможности роботов первого поколения существенно
ограничены несовершенством их системы управления, служащей для реализации лишь
той жесткой программы, которая зафиксирована с помощью штекерной панели,
<<программируемого барабана>> или концевых выключателей. При изменении
параметров окружающей среды такой робот не может сам адаптироваться к новым
условиям, и требуется вмешательство человека"=оператора, чтобы корректировать
заданную ему программу, приспосабливая ее к другим параметрам. Таким образом,
можно сказать, что роботы первого поколения имеют только нижний уровень системы
управления "--- исполнительный. Поскольку конструктивно этот уровень управления
непосредственно связан с манипулятором, становится понятным и еще одно название
роботов первого поколения "--- программные манипуляторы.

\subsection{Роботы второго поколения}

Принципиальная невозможность автономного функционирования роботов первого
поколения, очень затруднявшая их внедрение в производство, привела ученых и
инженеров к настойчивым попыткам устранить этот недостаток. Их поиски путей
обеспечения роботу большей свободы действий увенчались в конце концов успехом: в
1967 г. в США (Станфордский университет) была создана первая модель роботов
нового типа "--- роботов второго поколения.

\textbf{Роботы второго поколения} "--- это очувствленные роботы, предназначенные
для работы с неориентированными объектами произвольной формы, осуществления
сборочных и монтажных операций, сбора информации о внешней среде.

Отличительная их черта от роботов первого поколения "--- действие в зависимости
от окружающей обстановки: при изменении параметров объекта манипулирования (его
углов ориентации, координат) и окружающей среды (скажем, конфигурации каких-либо
препятствий на пути следования <<руки>>) они могут скорректировать свои действия
в соответствии с этими изменениями. Такая гибкость <<поведения>> роботов второго
поколения достигается благодаря тому, что они <<очувствлены>>, т. е. их система
управления снабжена большим числом разнообразных сенсорных датчиков, как внешних
(телевизионные, локационные, тактильные), так и внутренних (датчиков положения
<<рук>> относительно корпуса робота, датчиков развиваемых манипулятором усилий и
моментов и т. п.). Кроме того, главным звеном системы управления
<<очувствленного>> робота служит ЭВМ, воспринимающая сигналы всех датчиков о
внешней среде и состоянии самого манипулятора. На основании этих сигналов ЭВМ с
помощью заранее заложенной в нее программы устанавливает закон управления
исполнительным механизмом робота с учетом изменений окружающей обстановки. На
рисунке ~\ref{fig:9} показан робот второго поколения, а на рисунке ~\ref{fig:10}
"--- различные варианты внешнего вида адаптивных систем управления.

\begin{figure}[h!]
    \centering
    \includegraphics[width=10cm]{9.png}
    \caption{\label{fig:9}%
    Робот второго поколения}
\end{figure}

\begin{figure}[h!]
    \centering
    \includegraphics[width=10cm]{10.png}
    \caption{\label{fig:10}%
    Системы управления роботами второго поколения}
\end{figure}

Работа робота второго поколения осуществляется по следующему алгоритму:
\begin{enumerate}
    \item информация с технических органов чувств по обратным связям поступает
    в управляющую систему; 
    \item управляющая система обрабатывает полученную информацию;
    \item управляющая система подает управляющие сигналы на исполнительные
    механизмы.
\end{enumerate}

Таким образом, очувствленные роботы способны распознавать <<ситуации>> и
автоматически приспосабливаться (адаптироваться) к заранее не определенным и
изменяющимся условиям эксплуатации, т.е. становиться \textbf{адаптивными роботами}.

\subsection{Роботы третьего поколения}

\textbf{Роботы третьего поколения} "--- это интеллектуальные роботы,
предназначенные не только и не столько для воспроизведения физических и
двигательных функций человека, сколько для автоматизации его интеллектуальной
деятельности, т.е. для решения интеллектуальных задач. 

Роботы третьего поколения существенно отличаются от своих предшественников. И не
только гораздо более сложным алгоритмическим обеспечением, требующим более
совершенной вычислительной техники, но и совсем другим уровнем сложности
сенсорной системы. В общем случае <<интеллектуальный>> робот должен обладать
способностью зрительно воспринимать окружающую обстановку, формировать модель
внешней среды, анализировать и распознавать сходные ситуации и даже понимать
язык и вести в устной форме диалог с человеком. Но и это еще не все.
Принципиально важная особенность этих роботов "--- способность самообучаться в
процессе решения поставленных человеком задач, что неограниченно расширяет их
возможности. На рисунке ~\ref{fig:11} показан робот третьего поколения \cite{11}.

\begin{figure}[h!]
    \centering
    \includegraphics[width=4cm]{11.png}
    \caption{\label{fig:11}%
    Лабораторный подвижный робот третьего поколения: 1 "--- привод колеса, 9, 2 "---
    опорные ролики, 3 "--- датчики контакта, 4 "--- устройство управления
    телевизионной камерой, 5 "--- система управления роботом, связанная с ЭВМ, 6 "---
    оптический дальномер, 7 "--- антенна радиосвязи, 8 "--- подвижная телевизионная
    камера}
\end{figure}

\section{Классификация роботов}

Существуют различные виды роботов, отличающиеся способами управления,
назначением и решаемым классам задач \cite{5}. 

\subsection{Классификация роботов по типу управления}
По методу управления, или степени непосредственного участия человека в
управлении, роботы подразделяются на три класса:

\begin{enumerate}
    \item биотехнические;
    \item интерактивные;
    \item автоматические.
\end{enumerate}

\textbf{Биотехнические роботы} функционируют только с непосредственным участием
человека"=оператора, который фактически берет на себя управление исполнительными
механизмами. В зависимости от способа реализации биотехнического управления
можно выделить дистанционно управляемые копирующие роботы (управление с помощью
задающего механизма), командные (управляются с кнопочного или клавишного
пульта), экзоскелетоны (управляются биоимпульсами) и полуавтоматические
(управляемые с помощью ЭВМ). роботы.

\textbf{Интерактивные роботы} в отличие от биотехнических имеют устройства
памяти для автоматического выполнения отдельных действий и могут управляться
попеременно оператором или автоматически. В зависимости от формы участия человека-оператора интерактивное
управление может быть следующих видов: автоматизированное (когда происходит чередование во времени
автоматических режимов управления с биотехническими), диалоговое, супервизорное (когда все части заданного цикла операций выполняются роботом поэтапно, но переход от одного этапа к следующему
осуществляется после подачи оператором соответствующей целеуказательной команды).

\textbf{Роботы с автоматическим управлением} могут полностью или частично
функционировать без участия оператора. К ним относятся: автооператоры
(непрограммируемые автоматические манипуляторы, т.е. устройства, выполняющие
цикл несложных действий по жестко заданной, неизменяемой программе) и автономные
роботы (могут функционировать вполне самостоятельно без непосредственного
участия в их управлении человека"=оператора. Как правило, это очувствленные
роботы с элементами искусственного интеллекта. К ним относят космические и
подводные роботы).

\subsection{Классификация роботов по принципу управления}

\begin{enumerate}
    \item жестко программируемые;
    \item адаптивные;
    \item гибко программируемые.
\end{enumerate}

\textbf{Жестко программируемые} "--- это такие роботы, программа действий
которых содержит полный набор информации, не изменяющейся в процессе работы,
несмотря на изменение внешних условий.

\textbf{Адаптивные роботы} имеют сенсорное обеспечение, позволяющее
корректировать программные действия в соответствии с получаемой информацией о
внешней среде и состоянии самого робота, т.е. приспосабливать свои действия к
изменению внешних условий.

\textbf{Гибко программируемые роботы} способны полностью формировать программу
своих действий на основе поставленной цели и получаемой информации об окружающей
среде.

\subsection{Классификация роботов по назначению и решаемому классу задач}

По назначению и решаемому классу задач роботы всех поколений могут быть
разделены на две большие группы:

\begin{enumerate}
    \item производственные;
    \item исследовательские.
\end{enumerate}

\textbf{Производственные роботы} предназначены для выполнения тяжелой,
монотонной, вредной и опасной для здоровья людей физической работы, а также
отдельных видов трудоемких, напряженных и утомительных умственных работ
(проектирование, информационное обеспечение, управление).

\textbf{Исследовательские роботы} "--- это роботы, предназначенные для поиска,
сбора, переработки и передачи информации об исследуемых объектах. Такими
объектами могут быть труднодоступные, а также недоступные для человека сферы
"--- космическое пространство, океанские глубины, недра Земли, эктремальные
лабораторные условия и т.п. Либо области, где требуются выявление, переработка и
анализ огромных количеств информации, например, информационный поиск и разведка,
искусство и литература \cite{11}.

\section{Сферы применения роботов}

\subsection{Медицина}

Одним из наиболее интересных направлений, где используется робототехника, является медицина. Здесь задействование передовых технологических достижений позволяет более эффективно и качественно спасать человеческие жизни. Использование роботов позволяет более продуктивно осуществлять сложнейшие хирургические операции, в том числе дистанционно. Кроме того, они участвуют в лечении многих болезней \cite{6}.

В медицине достигнут большой прорыв с тех пор, как стали использоваться бионические протезы, которыми человек может управлять при помощи собственной нервной системы.После ампутации конечности в организме остаются двигательные нервы, и хирург прикрепляет их остатки к небольшому участку крупной мышцы. Например, если была утрачена рука, нервы перемещают в область грудной мыщцы.

Далее происходит самое интересное: человек хочет вытянуть руку, мозг направляет сигнал мышце с присоединенным нервом. Электроды фиксируют сигнал и отправляют импульс по проводам в процессор внутри протеза руки.

Более того, при помощи протеза человек может чувствовать прикосновение, тепло и давление \cite{7}.


\subsection{Космос}

Космические роботы "--- это роботы, используемые в космосе. Этот тип будет включать в себя роботы, используемые на Международной космической станции, Canadarm, которая использовалась в <<Шаттлах>>, а также марсоходы и другие машины, используемые в космосе. Данные технологии помогают человечеству осваивать космос, и имеют возможность работать там, где человек по своим физиологическим причинам не сможет этого делать, например, в космосе есть радиация, а также отсутствует воздух, поэтому человек не может выйти в открытый космос без скафандра, чего нельзя сказать про робота.

Для работы на различных поверхностях были разработаны различные модели роботов, так, например, для работы на поверхность луны для работы были отправлены следующие луноходы.

\textbf{Луноход"=1} "--- первый в мире дистанционно-управляемый самоходный аппарат, успешно работавший на Луне. 

\textbf{Луноход"=2} "--- второй в мире дистанционно-управляемый самоходный аппарат. Он был разработан для фотосъёмки и видеосъёмки Луны, проведения экспериментов с наземным лазерным дальномером и прочих операций.

Так же нельзя не упомянуть другой знаменитый и не маловажный космический робот"=телескоп "--- Хаббл.

О космическом телескопе Хаббл, запущенном НАСА и ЕКА, знают все. Это целая космическая обсерватория с большим количеством аппаратуры, позволяющей получать снимки отдаленных уголков Вселенной в разных диапазонах длин волн. Станция названа в честь известного астронома Эдвина Хаббла.

Cуществует ещё огромное множество космических роботов, который позволяют исследовать космические просторы, а главное сделать более безопасными для человеческих жизней. Потому в настоящее время занятие робототехникой можно считать перспективным направлением \cite{8}.

\subsection{Быт}

Большую популярность набирают бытовые роботы, предназначенные для помощи человеку в его повседневной жизни \cite{9}.

\begin{enumerate}
    \item \textbf{Помощники в быту:}
    \begin{enumerate}
        \item робот"=газонокосилка ухаживает за садом или дачным участком;
        \item робот"=снегоуборщик "--- уникальное самоходное устройство на гусеничном блоке, оснащенное ковшом для снега;
        \item робот"=пылесос выполняет уборку дома или квартиры;
        \item робот"=мойщик окон автоматически чистит стекла на окнах, зеркала в доме, витрины магазинов. Убирает труднодоступные и очень высокие окна, в том числе на высоких этажах;
        \item робот для чистки бассейна очищает поверхность бассейна, одновременно фильтруя воду.
    \end{enumerate}
    \item \textbf{Развлечения и охрана.} Это всевозможные электронные питомцы, которые развлекают пользователя и выполняют охранную функцию.
    \item \textbf{Продвинутые помощники.} Это последние модели человекоподобных роботов. Они носят вещи, выполняют несложные задачи по дому, охраняют жилище.
\end{enumerate}

\subsection{Производство}

Современные заводы и предприятия далеко продвинулась за счет современных технологий. Автоматизированные промышленные роботы применяются для сварки, укладки, покраски и прочих операций, требующих многократного повторения и высокой точности.

Чаще всего такие механические работники представляют собой механизм, напоминающий человеческую руку. Обычно это универсальное устройство с несколькими осями подвижности и фланцем для закрепления рабочего инструмента.

Использование промышленных роботов значительно увеличивает производительность, в то время как человеческие ресурсы освобождаются для более важных задач \cite{7}.

\subsection{Военное дело}

Боевой робот (военный робот) "--- устройства автоматики, заменяющее человека в боевых ситуациях для сохранения человеческой жизни или для работы в условиях, несовместимых с возможностями человека, в военных целях: разведка, боевые действия, разминирование и т.п.

При этом теоретически больше всего нуждаются в использовании роботов сухопутные войска, как наиболее <<контактные>> и несущие наибольшие потери в войне любого типа. На практике они отстают в этом от авиации, тем не менее, развитие безэкипажной техники идет достаточно быстро.

На сегодняшний день существует невероятное количество военных роботов, которые предназначаются для наземного сражения. Как правило, их подразделяют на четыре масштабные группы:
\begin{enumerate}
    \item Разведывательные;
    \item Инженерные роботы;
    \item Роботы для боя;
    \item Роботы для работы в тылу.
\end{enumerate}

Конечно, не все автоматизированные аппараты можно подразделять по данным критериям. Такие роботы характеризуются своими унифицированными платформами, с различными модулями. В результате этого любого робота можно превратить в другого военного робота \cite{10}.

\conclusion

Робототехника в современном мире является достаточно важной его частью. Это сфера, совершенствованию которой уделяется особое внимание ввиду пользы, которую она приносит человечеству. Промышленность, медицина, военно-промышленный комплекс, сельское хозяйство – лишь немногие примеры сфер, где робототехнические механизмы нашли обширное применение. Однако робототехническая отрасль, несмотря на все достижения, не является совершенной, и имеет массу проблем самого разного плана.

\bibliographystyle{gost780uv}
\bibliography{my}

\end{document}
