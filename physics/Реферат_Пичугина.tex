\documentclass[bachelor, och, referat]{SCWorks}

\usepackage{graphicx}

\usepackage[sort,compress]{cite}
\usepackage{amsmath}
\usepackage{amssymb}
\usepackage{amsthm}
\usepackage{fancyvrb}
\usepackage{longtable}
\usepackage{array}
\usepackage[english,russian]{babel}
\usepackage{tempora}
\usepackage{minted}
\usepackage[colorlinks=true, citecolor=black, filecolor=black, linkcolor=black,
urlcolor=black]{hyperref}
%\usepackage{times}



\begin{document}

% Кафедра (в родительном падеже)
\chair{информатики и программирования}

% Тема работы
\title{Главный парадокс черных дыр}

% Курс
\course{1}

% Группа
\group{151}

% Факультет (в родительном падеже) (по умолчанию "факультета КНиИТ")
%\department{факультета КНиИТ}

% Специальность/направление код - наименование
%\napravlenie{02.03.02 "--- Фундаментальная информатика и информационные технологии}
%\napravlenie{02.03.01 "--- Математическое обеспечение и администрирование информационных систем}
%\napravlenie{09.03.01 "--- Информатика и вычислительная техника}
\napravlenie{09.03.04 "--- Программная инженерия}
%\napravlenie{10.05.01 "--- Компьютерная безопасность}

% Для студентки. Для работы студента следующая команда не нужна.
\studenttitle{студентки}

% Фамилия, имя, отчество в родительном падеже
\author{Пичугиной Дарьи Николаевны}

% Заведующий кафедрой
\chtitle{к.\,ф.-м.\,н.} % степень, звание
\chname{С.\,В.\,Миронов}

%Научный руководитель (для реферата преподаватель проверяющий работу)
\satitle{доцент, к.\,ф.-м.\,н.} %должность, степень, звание
\saname{В.\,В.\,Машников}

% Руководитель практики от организации (только для практики,
% для остальных типов работ не используется)
%\patitle{к.\,ф.-м.\,н., доцент}
%\paname{Д.\,Ю.\,Петров}

% Семестр (только для практики, для остальных
% типов работ не используется)
%\term{2}

% Наименование практики (только для практики, для остальных
% типов работ не используется)
%\practtype{учебная}

% Продолжительность практики (количество недель) (только для практики,
% для остальных типов работ не используется)
%\duration{2}

% Даты начала и окончания практики (только для практики, для остальных
% типов работ не используется)
%\practStart{01.07.2016}
%\practFinish{14.07.2016}

% Год выполнения отчета
\date{2022}

\maketitle

% Включение нумерации рисунков, формул и таблиц по разделам
% (по умолчанию - нумерация сквозная)
% (допускается оба вида нумерации)
%\secNumbering


\tableofcontents



% Раздел "Определения". Может отсутствовать в работе
%\definitions

% Раздел "Определения, обозначения и сокращения". Может отсутствовать в работе.
% Если присутствует, то заменяет собой разделы "Обозначения и сокращения" и "Определения"
%\defabbr

\intro
В реферате речь будет идти о парадоксе исчезновения информации в чёрной дыре.
Вселенная "--- удивительное и странное место, наполненное необъяснимыми явлениями. Одно из таких явлений "--- информационный парадокс черных дыр, "--- кажется, нарушает основополагающий физический закон. Горизонт событий черной дыры считается последним рубежом: попав за его пределы, ничто не может покинуть черную дыру, даже свет. Но касается ли это информации? Будет ли она навсегда утеряна в черной дыре, как и все остальное?

Сама информация представляет собой специфическое расположение определенного количества частиц. Согласно квантовой механике, если бы мы смогли отслеживать все атомы, частицы и волны излучения, то мы теоретически могли бы видеть всю историю нашей Вселенной. Поэтому, если мы не перепишем квантовую механику и общую теорию относительности, эта информация должна продолжать существовать внутри черной дыры. Но, как показал Стивен Хокинг тем, что теперь называется излучением Хокинга, черные дыры испаряются и сбрасывают свою массу в течение невероятно длительных периодов времени. Эта потеря массы означала бы, что информация теряется навсегда. Выводы Хокинга были настолько противоречивыми, что его коллегам"=ученым потребовалось некоторое время, чтобы принять их и признать их важность, в конечном итоге назвав его информационным парадоксом черной дыры.

Прежде чем переходить к сути парадокса, рассмотрим основные понятия и принципы, на которые он опирается.

\section{Черные дыры}
\textbf{Черная дыра} "--- это область пространства, или, скорее, таинственный объект в центре области пространства, внутри которой сила тяжести настолько сильна, что ничто, даже свет, не может вырваться.

\begin{figure}[h!]
    \centering
    \includegraphics[width=10cm]{1.png}
\end{figure}

Впервые предположение о существовании черных дыр было высказано еще в конце 1700"=х годов. Однако именно Карл Шварцшильд (1873"=1916) и малоизвестный Йоханнес Дросте независимо друг от друга разработали современную идею черной дыры. Используя общую теорию относительности Эйнштейна, они обнаружили, что материя, сжатая до точки (возможно, не больше планковской длины), будет заключена в сферическую область пространства, из которой ничто не сможет вырваться. Предел этой области называется <<\textbf{горизонтом событий}>>, что означает невозможность наблюдения какого-либо события, происходящего внутри нее (поскольку информация не может выйти наружу).

Для невращающейся черной дыры радиус горизонта событий известен как радиус Шварцшильда и обозначает точку, в которой скорость выхода из черной дыры равна скорости света. Теоретически, любая масса может быть достаточно сжата, чтобы образовать черную дыру. Единственное требование "--- чтобы ее физический размер был меньше радиуса Шварцшильда. Например, наше Солнце станет черной дырой, если его масса будет заключена в сферу диаметром около 2,5 км. Наша Земля должна быть сжата до размера менее 1,77 см в диаметре.

Внутри горизонта событий находится сердце черной дыры. В классические времена, до появления квантовой механики, ее называли <<\textbf{сингулярностью}>>. Считалось, что материя, составляющая черную дыру, будет сжата до бесконечно малого размера, что даст бесконечно плотный объект. Однако мы не знаем, какую форму принимает материя при таких экстремальных плотностях, и, следовательно, насколько большой может быть черная дыра в пределах горизонта событий.

Интересная дилемма для астрофизиков заключается в том, что физические условия вблизи сингулярности приводят к полному нарушению законов физики. Однако в общей теории относительности нет ничего, что могло бы остановить существование изолированных, или <<голых>>, сингулярностей. Чтобы избежать ситуации, в которой мы могли бы наблюдать нарушение физики, была предложена гипотеза космической цензуры. Она гласит, что каждая сингулярность должна иметь горизонт событий, который скрывает ее от глаз "--- именно то, что мы наблюдаем у черных дыр.

Согласно классической теории общей теории относительности, после создания черной дыры она будет существовать вечно, так как ничто не может ее покинуть. Однако если принять во внимание квантовую механику, то окажется, что все черные дыры в конечном итоге испарятся, поскольку они медленно пропускают \textbf{излучение Хокинга}. Это означает, что время жизни черной дыры зависит от ее массы, причем маленькие черные дыры испаряются быстрее, чем большие. Например, для испарения черной дыры массой в 1 солнечную массу требуется 1067 лет (гораздо больше, чем текущий возраст Вселенной), в то время как черная дыра массой всего 1011 кг испарится в течение 3 миллиардов лет ~\cite{1}.

\section{Излучение Хокинга}
\subsection{Что такое излучение Хокинга?}

\textbf{Излучение Хокинга} описывает гипотетические частицы, образованные границей черной дыры. Это излучение предполагает, что черные дыры имеют температуру, обратно пропорциональную их массе. Другими словами, чем меньше черная дыра, тем горячее она должна светиться.

Хотя оно никогда не наблюдалось непосредственно, излучение Хокинга "--- это предположение, подтвержденное комбинированными моделями общей теории относительности и квантовой механики. Если доказать, что это действительно так, то излучение Хокинга будет означать, что черные дыры могут излучать энергию и, следовательно, уменьшаться в размерах, при этом самые маленькие из этих безумно плотных объектов быстро взрываются в пылу жара (а самые большие медленно испаряются в течение триллионов лет) ~\cite{3}.

\subsection{Почему черные дыры должны светиться?}

Когда материя попадает в черную дыру, она оказывается запертой от остальной части Вселенной. Это также устраняет определенную степень беспорядка; эту характеристику физики называют энтропией. Поскольку такое удаление материи делает Вселенную менее неупорядоченной, считалось, что это нарушает второй закон термодинамики.

Студент"=физик из Принстона в США Джейкоб Бекенштейн отметил, что граница вокруг пространства, наиболее подверженного воздействию безумной гравитации черной дыры "--- <<поверхность>>, называемая горизонтом событий, "--- должна увеличиваться по площади всякий раз, когда туда попадает материя. Он показал, что эта площадь представляет собой меру энтропии, которая в противном случае была бы потеряна, и это предложение должно разрешить парадокс.

Хокинг не был так уверен. \textbf{Энтропия} "--- это другой способ описания тепловой энергии, которая обязательно излучает радиацию. Если горизонт событий обладает энтропией, он должен каким"=то образом светиться, а значит, черные дыры не будут такими уж черными. Пытаясь опровергнуть абсурдное предположение Бекенштейна, Хокинг обсудил его с другими физиками и попытался показать с помощью математических моделей, что это невозможно. Вместо этого он обнаружил, что черные дыры действительно светятся холодным светом.

\subsection{Как черные дыры производят излучение Хокинга?}

Физический процесс, стоящий за испусканием частиц вблизи горизонта событий черной дыры, довольно сложен и основан на глубоком понимании математики квантовой теории поля. Обычно его описывают как результат разделения под действием гравитации парных <<виртуальных>> частиц, которые естественным образом возникают из вакуума. Обычно они рекомбинируют и аннулируются, но в данном случае разделение оставляет одну половину каждой пары для выхода в виде реального излучения.

\begin{figure}[h!]
    \centering
    \includegraphics[width=10cm]{2.png}
\end{figure}

На самом деле, популярное объяснение математики Хокинга описывает мимолетные виртуальные частицы, на которые воздействует экстремальная гравитация, при этом одна половина пары удаляет массу из черной дыры благодаря экстремальной гравитации, обеспечивающей частицу отрицательной энергией. Другие физики считают, что такое <<локализованное>> описание частиц, расщепляющихся по воображаемой линии, вводит в заблуждение.

\begin{figure}[h!]
    \centering
    \includegraphics[width=9cm]{3.png}
\end{figure}

Хотя нам нужна полная теория роли гравитации в квантовой механике, чтобы правильно описать это взаимодействие, выводы Хокинга показывают, как искривленное пространство может нарушить смесь квантовых свойств в полях вблизи горизонта событий, до такой степени, что черные дыры <<рассеивают>> некоторые свойства, оставляя другие нетронутыми. Именно эти нетронутые свойства напоминают определенные температуры излучения и могут вызвать сжатие черной дыры.

\section{Принцип сохранения информации}

Идея о том, что информация может быть сохранена, может показаться многим из нас несколько нелепой. Под <<сохранением>> мы подразумеваем, что во Вселенной существует фиксированное количество информации, и что, хотя она может быть изменена, ее нельзя ни создать, ни уничтожить. Это никак не согласуется с общепринятым мнением о том, что информации гораздо больше, чем раньше, и что она создается с постоянно растущей скоростью, а также с тем, что информация может быть потеряна в результате разрушения библиотек и архивов, а также распада и устаревания форматов хранения.

И все же идея сохранения информации принимается как данность в физических науках. Если информация является частью физической вселенной, как это принято считать во все более распространенной точке зрения, то разумно предположить, что она будет сохраняться, как и другие фундаментальные физические величины, такие как энергия или импульс. Это противоречие интригует и проливает свет на различия и возможную взаимосвязь между концепцией информации в физическом и социальном мире.

Учитывая, что любое изменение в физическом мире может быть представлено в абстрактных терминах двумя состояниями (до и после) со стрелкой, показывающей, что второе состояние развилось непосредственно из первого, мы имеем: <<самый фундаментальный из всех физических законов "--- сохранение информации. Сохранение информации "--- это просто правило, согласно которому каждое состояние имеет одну стрелку на входе и одну стрелку на выходе. Это гарантирует, что вы никогда не потеряете счет тому, с чего начали. Сохранение информации "--- это не обычный закон сохранения>>.

Эта идея лежит в основе одного из величайших противоречий современной физики "--- <<\textbf{парадокса черной дыры}>> Стивена Хокинга. Как только что"=либо попадает в черную дыру, оно исчезает из нашей Вселенной; масса черной дыры увеличивается, но это ничего не говорит нам о природе того, что туда попало; кажется, что она и информация, которую она несет, исчезли из Вселенной. Хокинг показал, что черные дыры испускают излучение, которое в конечном итоге приведет к рассеиванию черной дыры, возвращая ее массу/энергию во Вселенную. Но, согласно Хокингу, природа этого излучения одинакова независимо от того, что попало в черную дыру; информация не возвращается во Вселенную и поэтому не сохраняется ~\cite{2}.

\section{Что же такое информационный парадокс черной дыры?}

Если бы вы могли смотреть на черную дыру в течение миллиардов лет, вы бы увидели все, что она когда"=либо собирала, прилипшее снаружи, как мухомор. Вы могли бы указать на что угодно, и теоретически вы могли бы определить квантовое состояние каждой частицы и фотона, которые попали в черную дыру. Поскольку им потребуется бесконечное количество времени, чтобы полностью исчезнуть, все в порядке. Их информация навсегда сохранилась на поверхности черной дыры. Они все абсолютно мертвы, но их информация, их драгоценная квантовая информация, находится в полной безопасности.

Если бы вы могли разгадать черную дыру, вы могли бы добраться до всей квантовой информации, описывающей все, что черная дыра когда"=либо поглощала. Но в 1975 году Хокинг понял, что у черных дыр есть температура, и в течение огромного периода времени они будут испаряться, пока не останется ничего, выпуская свою массу и энергию обратно во Вселенную. Это явление известно как излучение Хокинга ~\cite{4}.

Но эта идея породила парадокс. Информация о том, что попало в черную дыру, сохраняется благодаря замедлению времени, но при этом сама масса черной дыры испаряется. В конце концов, она полностью исчезнет, и куда тогда денется наша информация? Та информация, которая не может быть уничтожена? Это озадачило астрономов. Они десятилетиями работали над решением этой проблемы. Здесь есть несколько интересных вариантов:

\begin{enumerate}
\item Черные дыры вообще не испаряются, и Хокинг был неправ.
\item Информация внутри черной дыры каким"=то образом выходит обратно, в то время как излучение Хокинга исчезает.
\item Черная дыра удерживает информацию до самого конца, а когда две последние частицы испаряются, вся информация внезапно высвобождается обратно во Вселенную.
\item Вся информация превращается в мельчайшие биты, и ничего не теряется ИЛИ информация сжимается в микроскопическое пространство, которое остается после испарения самой черной дыры ~\cite{5}.
\end{enumerate}

И, возможно, физики никогда не найдут ответ на этот вопрос. Хокинг предложил новую идею для разрешения информационного парадокса черной дыры. Он предположил, что существует способ, при котором новое излучение Хокинга может быть запечатлено информацией новой материи, падающей в черную дыру. Таким образом, информация всего, что попадает внутрь сохраняется исходящим излучением, возвращаясь во Вселенную и разрешая парадокс. Это лишь предположение, поскольку само излучение Хокинга никогда не было обнаружено. Мы не знаем, в правильном ли направлении движемся, и есть ли вообще способ разрешить парадокс.

В подобных ситуациях нам напоминают, как мало мы на самом деле понимаем о Вселенной. Некоторые аспекты нашего понимания всего этого процесса неясны, и потребуется гораздо больше детективной работы и экспериментов, чтобы приблизиться к истине.

\conclusion

В этом парадоксе интересно то, что все возможные ответы ведут к новой физике. В любом случае, если мы разрешим парадокс, мы узнаем что"=то новое о Вселенной.

Не ждите, что эта 40"=летняя загадка будет решена в ближайшее время. И решение, скорее всего, придет от молодого физика, о котором вы никогда не слышали, или от человека, который еще не родился.

\bibliographystyle{gost780uv}
\bibliography{my}



\end{document}
